\documentclass[10pt,a4paper]{article}
\usepackage[utf8x]{inputenc}
\usepackage{ngerman}
\usepackage{ucs}
\usepackage[fleqn]{amsmath}
\usepackage{amsfonts}
\usepackage{amssymb}
\usepackage{fancyhdr}
\usepackage{totpages}
\usepackage{cancel} % Kürzen von Brüchen

%\usepackage{breqn} 
\usepackage{graphicx}
\usepackage{pdfpages}

\author{Alina Bombeck, Pedram Mehdizadeh Taheri, Sebastian Döbel}
\title{}

% *** Seitenränder ************************
\usepackage[left=2.5cm,right=2.0cm,bottom=30mm,top=30mm, headsep=1.5cm]{geometry}	

% Kopf- und Fußzeilen
\pagestyle{fancy}
\lhead{Alina Bombeck \\ Pedram Mehdizadeh Taheri \\  Sebastian Döbel}
\chead{Formale Grundlagen der Informatik - Übungsblatt 2}
\rhead{4. November 2014}
\rfoot{Seite \thepage \ von \ref{TotPages}}
\cfoot{}

% paar shortcuts
\newcommand{\IN}{\mathbb{N}}
\newcommand{\f}[2]{\frac {#1} {#2}}
\renewcommand{\o}{\omega}

\begin{document}

\section*{Aufgabe 3.3}
\subsection*{Aufgabe 3.3.1.}
\[L(A_1)=(a^* +ba^*b)^* (ba^*+\epsilon)\]
\[L(A_2)=(a^*+ba^*b)^* ba^* \]
\[L^\o(A_1) = (a^*ba^*b)^\o +ba^\o\]
\[L^\o(A_2) = a^*((a^*ba^*b)^\o + ba^\o) \]

\subsection*{Aufgabe 3.3.2}
Hier muss noch das lustige Bild rein.

\subsection*{Aufgabe 3.3.3}
\[L(A_3)=(a^*+ba^*b)^*ba^*\]
\[L^\o(A_3)=(a^*+ba^*b)^*ba^\o\]
\[L(A_3) = L(A_1)\cap L(A_2) \]
\[L^\o(A_3) \neq L^\o(A_1)\cap  L^\o(A_2) = (a^*ba^*b)^\o + ba^\o\] 

\subsection*{Aufgabe 3.3.4}
Hier kommt auch son lustiges Bild vom Handy hin.

\subsection*{Aufgabe 3.3.5}

\[L(A_4) \neq L(A_1)\cap L(A_2) = \text{HIER MUSS NOCH DER REGULÄRE AUSDRUCK ERZEUGT WERDEN}\]
\[L^\o(A_4) = = (a^*ba^*b)^\o + ba^\o = L^\o(A_1)\cap L^\o(A_2)\]


\newpage
\section*{Aufgabe 3.4}

\subsection*{TS1 \& TS2:}
Die Transaktionssysteme sind nicht bisimilar, da kein c-Übergang von $P_{1}$ nach $P_{3}$ vorhanden ist.

\subsection*{TS1 \& TS3:}
Die Transaktionssysteme sind nicht bisimilar, da kein c-Übergang von $P_{1}$ nach $P_{3}$ vorhanden ist.

\subsection*{TS1 \& TS4:}
Die Transaktionssysteme sind nicht bisimilar, da kein c-Übergang von $P_{1}$ nach $P_{3}$ vorhanden ist.

\subsection*{TS2 \& TS3:}
Die Transaktionssysteme sind bisimilar. Die Bisimilaritätsrelation lautet:

$B = \{(Q_{0},R_{0}),(Q_{1},R_{1}),(Q_{1},R_{2}),(Q_{2},R_{3}),(Q_{2},R_{4}),(Q_{3},R_{0})\}$

\subsection*{TS2 \& TS4:}
Die Transaktionssysteme sind nicht bisimilar, da kein c-Übergang von $S_{2}$ nach $P_{4}$ vorhanden ist.

\subsection*{TS3 \& TS4:}
Die Transaktionssysteme sind nicht bisimilar, da kein c-Übergang von $S_{2}$ nach $S_{4}$ vorhanden ist.


\end{document}